\chapter{Por que estudar Inferência Causal?}
\label{cap:intro}

Você já deve ter ouvido diversas vezes que
\textbf{correlação não implica causalidade}. Contudo,
o que é causalidade e como 
ela pode ser usada para resolver problemas práticos?
Antes de estudarmos definições formais,
veremos como conceitos intuitivos de causalidade
podem ser necessários para resolver questões
usuais em Inferência Estatística.
Para tal, a seguir estudaremos um exemplo de \citet{Glymour2016}.

\section{O Paradoxo de Simpson}
\label{sec:simpson}

\begin{table}
\begin{knitrout}
\definecolor{shadecolor}{rgb}{0.969, 0.969, 0.969}\color{fgcolor}\begin{kframe}
\begin{verbatim}
##     C   0   1
## Z T          
## 0 0    36 234
##   1     6  81
## 1 0    25  55
##   1    71 192
\end{verbatim}
\end{kframe}
\end{knitrout}
 \caption{Tabela de frequência conjunta 
 das variáveis binárias $T$, $C$, e $Z$.}
 \label{tabs:simpson}
\end{table}

Considere que observamos em $500$ pacientes $3$ variáveis: 
$T$ e $C$ são as indicadoras de que, respectivamente,
o paciente recebeu um tratamento e o paciente curou de uma doença, e
$Z$ é uma variável binária cujo significado será discutido mais tarde.
Os dados foram resumidos na \cref{tabs:simpson}.

Em uma primeira análise desta tabela, podemos 
verificar a efetividade do tratamento 
dentro de cada valor de $Z$.
Por exemplo, quando $Z=0$,
a frequência de recuperação dentre aqueles que 
receberam e não receberam o tratamento 
são, respectivamente: $\frac{81}{6+81} \approx 0.93$ e
$\frac{234}{36+234} \approx 0.87$.
Similarmente, quando $Z=1$,
as respectivas frequências são:
$\frac{192}{71+192} \approx 0.73$ e
$\frac{55}{25+55} \approx 0.69$.
À primeira vista, para todos os valores de $Z$, a taxa de recuperação 
é maior com o tratamento do que sem ele.
Isso nos traz informação de que 
o tratamento é efetivo na recuperação do paciente?

Em uma segunda análise, podemos considerar
apenas as contagens para as variáveis $T$ e $C$,
sem estratificar por $Z$.
Dentre os pacientes que receberam e não receberam o tratamento
as taxas de recuperação são, respectivamente:
$\frac{81+192}{6+71+81+192} \approx 0.78$ e
$\frac{234+55}{36+25+234+55} \approx 0.83$. Isto é,
sem estratificar por $Z$, a frequência de recuperação é
maior dentre aqueles que não receberam o tratamento
do que dentre aqueles que o receberam.

O que é possível concluir destas análises?
Uma conclusão ingênua poderia ser a de que,
se $Z$ não for observada, então o tratamento não é recomendado.
Por outro lado, se $Z$ é observada, 
não importa qual seja o seu valor, o tratamento será recomendado.
A falta de sentido desta conclusão ingênua é
o que tornou este tipo de dado famoso como sendo
um caso de Paradoxo de Simpson \citep{Simpson1951}.

Contudo, se a conclusão ingênua é paradoxal e incorreta,
então qual conclusão pode ser obtida destes dados?
A primeira lição que verificaremos é que não é possível obter
uma conclusão sobre o \textbf{efeito causal} do tratamento usando
apenas a informação na tabela, isto é, associações.
Para tal, analisaremos a tabela dando 
dois nomes distintos para a variável $Z$.
Veremos que, usando exatamente os mesmos dados,
uma conclusão válida diferente é obtida para cada nome de $Z$.
Em outras palavras, o efeito causal depende de
mais informação do que somente aquela disponível na tabela.

Em um primeiro cenário, considere que $Z$ é a indicadora de que
o sexo do paciente é masculino.
Observando a tabela, notamos que, proporcionalmente, 
mais homens receberam o tratamento do que mulheres.
Como o tratamento não tem qualquer influência sobre o sexo do paciente,
podemos imaginar um cenário em que, proporcionalmente,
mais homens escolheram receber o tratamento do que mulheres.

Usando esta observação,
podemos fazer sentido do Paradoxo anteriormente obtido.
Quando agregamos os dados,
notamos que o primeiro grupo de pacientes que 
receberam o tratamento é predominantemente composto por homens e,
similarmente, o segundo grupo de 
pacientes que não receberam o tratamento é
predominantemente composto por mulheres.
Isto é, na análise dos dados agregados estamos 
essencialmente comparando a taxa de recuperação de homens 
que receberam o tratamento com
a de mulheres que não receberam o tratamento.
Se assumirmos que, independentemente do tratamento, 
mulheres tem uma probabilidade de recuperação maior do que homens,
então a taxa de recuperação menor no primeiro grupo pode ser explicada
pelo fato de ele ser composto predominantemente por homens e
não pelo fato de ser o grupo de pacientes que recebeu o tratamento.
Também, da análise anterior, obtemos que para cada sexo, 
a taxa de recuperação é maior com o tratamento  do que sem ele. 
Isto é, neste cenário, o tratamento parece efetivo para
a recuperação dos pacientes.
Isto significa que a análise estratificando $Z$ é sempre a correta?

Caso o significado da variável $Z$ seja outro, veremos que 
esta conclusão é incorreta.
Considere que $Z$ é a indicadora de que 
a pressão sanguínea do paciente está elevada.
Além disso, é sabido que o tratamento tem como
efeito colateral aumentar o risco de
pressão elevada nos pacientes.
Neste caso, o fato de que
há mais indivíduos com pressão elevada dentre aqueles que
receberam o tratamento é
um efeito direto do tratamento.

Usando esta observação,
podemos chegar a outras conclusões sobre o
efeito do tratamento sobre a recuperação dos pacientes.
Para tal, considere que o tratamento tem
um efeito positivo moderado sobre a recuperação dos pacientes,
mas que a pressão sanguínea elevada prejudica gravemente a recuperação.
Quando fazemos comparações apenas dentre indivíduos com pressão alta ou
apenas dentre indivíduos sem pressão alta, 
não é possível identificar o efeito coletaral do tratamento.
Isto é, observamos apenas
o efeito positivo moderado que o tratamento tem sobre a recuperação.
Por outro lado, quando fazemos a análise agregada,
observamos que a frequência de recuperação é 
maior dentre os indivíduos que não receberam o tratamento
do que dentre os que o receberam.
Isso ocorre pois o efeito colateral negativo tem um impacto
maior sobre a recuperação do paciente do que o efeito geral benéfico.
Assim, neste cenário, o tratamento não é
eficiente para levar à recuperação do paciente.

Como nossas conclusões dependem de qual história adotamos,
podemos ver que a mera apresentação da tabela é
insuficiente para determinar a eficiência do tratamento.
Observando com cuidado os cenários,
identificamos uma explicação geral para
as diferentes conclusões.
No primeiro cenário, quando $Z$ é sexo,
$Z$ é uma causa do indivíduo receber ou não o tratamento.
Já no segundo cenário, quando $Z$ é pressão elevada,
o tratamento é causa de $Z$. Isto é,
a diferença nas relações entre as variáveis
explica as diferenças entre as conclusões obtidas.

Ao longo do curso, desenvolveremos ferramentas para
formalizar a diferença entre estes cenários e, com base nisso,
conseguir estimar o efeito causal que uma variável $X$ tem sobre outra variável $Y$.
Contudo, para tal, será necessário desenvolver
um modelo em que seja possível descrever relações causais.
Esta questão será tratada no \cref{cap:dag}.

\subsection{Exercícios}

\begin{exercise}[{\citet{Glymour2016}[p.6]}]
 Há evidência de que há correlação positiva entre
 uma pessoa estar atrasada e estar apressada.
 Isso significa que uma pessoa pode evitar atrasos
 se não tiver pressa? 
 Justifique sua resposta em palavras.
\end{exercise}
