\documentclass[11pt]{scrbook}
\usepackage[portuguese]{babel}
\usepackage[utf8]{inputenc}

\usepackage{nameref}
\usepackage[colorlinks=true, urlcolor=blue, citecolor=blue, linkcolor=blue]{hyperref}
\usepackage{crossreftools}
\pdfstringdefDisableCommands{%
    \let\Cref\crtCref
    \let\cref\crtcref
}

\usepackage{amsfonts, amsmath, amssymb, amsthm, thmtools}
  
\usepackage[norsk,nameinlink]{cleveref}

\declaretheorem[name=Teorema, refname={Teorema, Teoremas}, Refname={Teorema, Teoremas}, parent=chapter]{theorem}
\declaretheorem[name=Algoritmo, refname={Algoritmo, Algoritmos}, Refname={Algoritmo, Algoritmos}, sibling=theorem, style=definition]{algorithm2}
\declaretheorem[name=Corolário, refname={Corolário, Corolários}, Refname={Corolário, Corolários}, sibling=theorem]{corollary}
\declaretheorem[name=Definição, refname={Definição, Definições}, Refname={Definição, Definições}, sibling=theorem, style=definition]{definition}
\declaretheorem[name=Exemplo, refname={Exemplo, Exemplos}, Refname={Exemplo, Exemplos}, sibling=theorem, style=definition]{example}
\declaretheorem[name=Lema, refname={Lema, Lemas}, Refname={Lema, Lemas}, sibling=theorem]{lemma}
\declaretheorem[name=Exercício, refname={Exercício, Exercícios}, Refname={Exercício, Exercícios}, sibling=theorem, style=definition]{exercise}

\crefname{section}{Seção}{Seções}
\Crefname{section}{Seção}{Seções}
\crefname{chapter}{capítulo}{capítulos}
\Crefname{chapter}{Capítulo}{Capítulos}
\crefname{table}{tabela}{tabelas}
\Crefname{table}{Tabela}{Tabelas}

\newcommand{\crefrangeconjunction}{ a }
\newcommand{\crefpairconjunction}{ e }
\newcommand{\creflastconjunction}{ e }

\setcounter{tocdepth}{4}

\usepackage{color, xcolor, enumitem, float, layout, multirow, setspace, slashbox, verbatim}
\usepackage{algorithm2e, algorithmicx, listings}
\usepackage{caption, subcaption}
\usepackage[authoryear]{natbib}
%\usepackage[all]{xy}
\usepackage{authblk}
\usepackage{cprotect}
\bibpunct[; ]{(}{)}{,}{a}{,}{;}

\usepackage{scrlayer-scrpage}
\pagestyle{scrplain}

\usepackage{graphicx}
\graphicspath{{figures/}{../figures/}}
\renewcommand{\floatpagefraction}{.9}

\usepackage{tikz}
\usetikzlibrary{positioning}
\definecolor{offwhite}{HTML}{F2EDED}
\tikzset{
    > = stealth,
    every node/.append style = {
        text = black
    },
    every path/.append style = {
        arrows = ->,
        draw = black,
        fill = black
    }
}

\usepackage{ifthen}
\newboolean{noShowSolutions}
\setboolean{noShowSolutions}{true}
\newcommand{\solutioncmd}[2]{{{#1}}{{#2}}}
\newcommand{\solution}[2]{\ifthenelse {\boolean{noShowSolutions}} {\solutioncmd{#2}}{\solutioncmd{#1} }}
\usepackage{xcolor}
\def\X{***ATTN***}
\newcommand{\red}[1]{\textbf{\color{red} ***ATTN*** #1}}
\renewcommand{\vec}[1]{\mathbf{#1}}

\def\limn{\lim_{n \rightarrow \infty}}
\def\convas{\stackrel{a.s.}{\longrightarrow}}
\def\convp{\stackrel{\P}{\longrightarrow}}

\DeclareRobustCommand{\rchi}{{\mathpalette\irchi\relax}}
\newcommand{\irchi}[2]{\raisebox{\depth}{$#1\chi$}}

\def\sA{{\mathcal A}}
\def\sE{{\mathcal E}}
\def\sG{{\mathcal G}}
\def\sV{{\mathcal V}}

\def\CM{{CM} }
\def\SCM{{SCM}}
\def\ACE{{ACE}}
\def\CACE{{CACE}}
\def\LACE{{LATE}}
\def\hACE{{\widehat{\ACE}}}
\def\hCACE{{\widehat{\CACE}}}
\def\hmu{{\widehat{\mu}}}
\def\hdoy{{\widehat{\E}}_{1}[Y|do(X=x)]}
\def\hdoyb{{\widehat{\E}}_{2}[Y|do(X=x)]}
\def\hdoyc{{\widehat{\E}}_{3}[Y|do(X=x)]}
\def\hdoyz{{\widehat{\E}}_{1}[Y|do(X=x),\Z]}
\def\hdoyzb{{\widehat{\E}}_{2}[Y|do(X=x),\Z]}
\def\hf{\widehat{f}}

\def\I{{\mathbb I}}
\def\P{{\mathbb P}}
\def\E{{\mathbb E}}
\def\U{{\mathbb U}}
\def\V{{\mathbb V}}
\def\N{{\mathbb N}}
\def\R{{\mathbb R}}
\def\seqn{{_{n \in N}}}
\def\bu{{\vec u}}
\def\bv{{\vec v}}
\def\bz{{\vec z}}
\def\W{{\vec{W}}}
\def\w{{\vec{w}}}
\def\X{{\vec{X}}}
\def\x{{\vec{x}}}
\def\Y{{\vec{Y}}}
\def\y{{\vec{y}}}
\def\Z{{\vec{Z}}}
\def\z{{\vec{z}}}


\newcommand{\ind}{\perp\!\!\!\!\perp}
\newcommand{\dsep}{\perp^d}

\usepackage{epigraph}


\makeatletter
\def\maxwidth{ %
  \ifdim\Gin@nat@width>\linewidth
    \linewidth
  \else
    \Gin@nat@width
  \fi
}
\makeatother

\definecolor{fgcolor}{rgb}{0.345, 0.345, 0.345}
\newcommand{\hlnum}[1]{\textcolor[rgb]{0.686,0.059,0.569}{#1}}%
\newcommand{\hlstr}[1]{\textcolor[rgb]{0.192,0.494,0.8}{#1}}%
\newcommand{\hlcom}[1]{\textcolor[rgb]{0.678,0.584,0.686}{\textit{#1}}}%
\newcommand{\hlopt}[1]{\textcolor[rgb]{0,0,0}{#1}}%
\newcommand{\hlstd}[1]{\textcolor[rgb]{0.345,0.345,0.345}{#1}}%
\newcommand{\hlkwa}[1]{\textcolor[rgb]{0.161,0.373,0.58}{\textbf{#1}}}%
\newcommand{\hlkwb}[1]{\textcolor[rgb]{0.69,0.353,0.396}{#1}}%
\newcommand{\hlkwc}[1]{\textcolor[rgb]{0.333,0.667,0.333}{#1}}%
\newcommand{\hlkwd}[1]{\textcolor[rgb]{0.737,0.353,0.396}{\textbf{#1}}}%

\usepackage{framed}
\makeatletter
\newenvironment{kframe}{%
 \def\at@end@of@kframe{}%
 \ifinner\ifhmode%
  \def\at@end@of@kframe{\end{minipage}}%
  \begin{minipage}{\columnwidth}%
 \fi\fi%
 \def\FrameCommand##1{\hskip\@totalleftmargin \hskip-\fboxsep
 \colorbox{shadecolor}{##1}\hskip-\fboxsep
     % There is no \\@totalrightmargin, so:
     \hskip-\linewidth \hskip-\@totalleftmargin \hskip\columnwidth}%
 \MakeFramed {\advance\hsize-\width
   \@totalleftmargin\z@ \linewidth\hsize
   \@setminipage}}%
 {\par\unskip\endMakeFramed%
 \at@end@of@kframe}
\makeatother

\definecolor{shadecolor}{rgb}{.97, .97, .97}
\definecolor{messagecolor}{rgb}{0, 0, 0}
\definecolor{warningcolor}{rgb}{1, 0, 1}
\definecolor{errorcolor}{rgb}{1, 0, 0}
\newenvironment{knitrout}{}{} % an empty environment to be redefined in TeX

\usepackage{alltt}
\IfFileExists{upquote.sty}{\usepackage{upquote}}{}

\textheight25cm \footskip0.5cm \topmargin-1.5cm \headsep0.2cm
\oddsidemargin-1.5cm \evensidemargin-1.5cm
\marginparwidth0cm \marginparsep = 0pt
\textwidth19cm
\onehalfspacing

\title{Inferência Causal \\[8mm]}
\subtitle{Notas de Aula \\[8mm]}
\author{Rafael Bassi Stern}
\date{Última revisão: \today \\[8mm]
Por favor, enviem comentários, typos e erros para rbstern@gmail.com}
